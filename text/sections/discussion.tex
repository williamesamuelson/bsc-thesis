\documentclass[../main.tex]{subfiles}
\graphicspath{{\subfix{../figures/}}}
\begin{document}
Something about the similarity of \cref{fig:diffde} to Haak's paper regarding critical damping.

The main takeaways from the implemented methods and the following simulations consist of two parts. Firstly, that the general evolution for the parallel quantum dot system at an EP indeed can be understood in terms of the sum of generalized modes given by \cref{eq:genmode}. This means that the Jordan blocks evolve separately, and that the initial condition decides what modes are involved in the dynamics of the system. Secondly, that the implemented methods give direct access to the generalized eigenvectors through numerical means, which in theory can be done for even larger systems and higher order EPs, unlike algebraic methods. The generalized eigenvectors can then be used as initial conditions, directly controlling what Jordan blocks are included in the evolution.

Applications and further theoretical outlook?
\end{document}
