\documentclass[../main.tex]{subfiles}
\graphicspath{{\subfix{../figures/}}}
\begin{document}

Summarizing the results of the thesis, a second order EP in the Liouvillian generating the dynamics in a parallel dot system was demonstrated. After finding the EP, Jordan decomposition was used to understand the evolution of the system in terms of the generalized modes, given by \cref{eq:dynep}. Intuitively, this means that the Jordan chains, consisting of generalized eigenvectors, evolve separately, and that the initial density matrix decides what modes are involved in the dynamics of the system. \Cref{eq:dynep} also implies that the dynamics at the EP for certain initial conditions includes algebraic decay, as opposed to pure exponential decay away from the EP. It was clear that these characteristics of the density operator also carry over to the current through the system. Furthermore, comparisons of the transient current at and away from the EP indicated similarities to Ref.~\cite{thermal} regarding critical damping at the EP, but for a system of quantum dots instead of a quantum thermal machine.

From a computational point of view, one may argue that the dynamics could just as well could have been simulated by a standard numerical integrator instead of the implemented methods in this thesis. However, the latter gives direct access to the generalized eigenvectors, which then could be used as initial conditions, directly controlling what modes to be included in the evolution. This increased the understanding of the dynamics of the system and might prove useful for probing the system for interesting and useful dynamics. It is also important to note that the generalized eigenvectors were calculated through numerical means, which in theory can be done for even larger systems and higher order EPs, unlike algebraic methods. %However, the numerical approach used can most likely be improved, perhaps by using Schur decomposition as described in~Ref. 

One of the main applications discussed in the literature on EPs is sensing technologies. As mentioned in the introduction, this is because eigenvalue splittings are more sensitive to small perturbations at an EP. In the framework of non-Hermitian physics, the eigenvalues of the operator in question, the Hamiltonian, are the energies of the system and hence directly observable. In Liouvillian physics, the eigenvalues instead correspond to the decay rates, which are more abstract and difficult to measure. An interesting open problem would be to find a useful observable that probes decay rates, such that similar sensing ideas could be applied to QDs and Liouvillian systems in general. Another proposed application for EPs is in quantum control, i.e., for steering a quantum system to a target state in an optimal way. The signs of critical decay at the EP in the parallel dot system are an indication of the validity of this application in systems of QDs. If the parallel dot system would act as a component in a bigger system, it may be important for it to reach its steady-state with critical damping. If so, EPs in quantum dot systems may serve as a piece in the puzzle of reaching optimal performance in future technologies.

To conclude, considering the sparse amount of literature regarding EPs in Liouvillian systems, and in particular systems involving QDs, it is interesting that a system of quantum dots indeed can give rise to EPs. This opens up many questions of how to understand and exploit EPs in QD systems in the future. This thesis begins to answer these questions and will hopefully lead to further work in the, so far, small world of exceptional points in quantum dots.

\end{document}
