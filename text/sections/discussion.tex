\documentclass[../main.tex]{subfiles}
\graphicspath{{\subfix{../figures/}}}
\begin{document}
In this thesis, we studied the dynamics of a parallel QD system at a second order EP, resulting in both theoretical and computational insights of EP physics for QD systems. To begin with, considering the sparse amount of literature regarding EPs in Liouvillian systems, and in particular systems involving QDs, it is interesting that a system of QDs indeed can give rise to EPs. After finding a second order EP for the parallel dot system, the theory of Jordan decomposition was applied, such that the evolution of the density operator was understood in terms of the generalized modes, given by \cref{eq:dynep}. Intuitively, this means that the Jordan chains, consisting of generalized eigenvectors, evolve separately, and that the initial density matrix decides what modes are involved in the dynamics of the system. \Cref{eq:dynep} also implies that the dynamics at the EP for certain initial conditions includes algebraic decay, as opposed to pure exponential decay away from the EP. It was clear that these characteristics of the density operator also carry over to the current through the system. Furthermore, comparisons of the transient current at and away from the EP were done, indicating similarities to Ref.~\cite{thermal} regarding critical decay towards the steady-state at the EP.

From a computational point of view, one may argue that the dynamics could just as well could have been simulated by a standard numerical integrator instead of the implemented methods in this thesis. However, the latter gives direct access to the generalized eigenvectors, which then could be used as initial conditions, directly controlling what modes to be included in the evolution. This increased the understanding of the dynamics of the system and might prove useful for probing the system for interesting and useful dynamics by varying the initial conditions. It is also important to note that the generalized eigenvectors were calculated through numerical means, which in theory can be done for even larger systems and higher order EPs, unlike algebraic methods. %However, the numerical approach used can most likely be improved, perhaps by using Schur decomposition as described in~Ref. 

One of the main applications discussed in the literature on EPs is in sensing technologies. This is because eigenvalues are more sensitive to small perturbations at an EP (source). In the framework of non-Hermitian physics, the eigenvalues of the operator in question, the Hamiltonian, are the energies of the system and hence directly observable. In Liouvillian physics, the eigenvalues instead correspond to the decay rates, which are more abstract and difficult to measure. An interesting open problem would be to find a useful observable that probes decay rates, such that similar sensing ideas could be applied to QDs and Liouvillian systems in general. Another proposed application for EPs is in quantum control, i.e., for steering a quantum state to a target state in an optimal way. This application is perhaps more clear in this thesis since \cref{fig:diffI} indicates a critical decay toward the steady-state. If a system of QDs would act as a component in a bigger system, it may be important for it to reach the steady state in an optimal way. If so, EPs probably will play a role in this problem. 

%Noise.

\end{document}
