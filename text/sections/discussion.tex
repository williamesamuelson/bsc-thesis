\documentclass[../main.tex]{subfiles}
\graphicspath{{\subfix{../figures/}}}
\begin{document}
From \cref{fig:minevsint}, it is clear that the implemented methods produce an accurate result at and away from the exceptional point. The relative distance to the numerical solver is on the order of $10^{-8}$, and with the use of strict tolerances of the numerical solver, this indicates an accurate method. Furthermore, when reducing the tolerances even more, the distance kept lowering, indicating that the numerical solver might be the limiting factor.

Since the dynamics may just as well be simulated by the numerical solver, the implemented methods are not that useful of themselves. The interesting things that instead can be taken from them are two things. Firstly, that the general evolution at an exceptional point can be understood in terms of the sum of generalized modes given by \cref{eq:genmode}. This means that the Jordan blocks evolve separately, and that the initial condition decides what Jordan blocks are involved in the dynamics of the system. Secondly, the implemented methods give direct access to the generalized eigenvectors through numerical means, which in theory can be done for even larger systems and higher order EPs, unlike algebraic methods. The generalized eigenvectors can then be used as initial conditions, directly controlling what Jordan blocks are included in the evolution.

Using the calculated generalized eigenvectors, this evolution in the different Jordan blocks could be evidenced in \cref{fig:diffrho0}. With an initial condition only overlapping the steady state and one other normal eigenvector, a pure exponential decay toward the steady state is seen by the straight lines in the log plot. When including both $\ket{\bar \rho}\rangle$ and $\ket{\rho_3}\rangle$ in the initial condition, the decay first follows the quicker decay channel, then gradually turns into following the decay of $\ket{\bar\rho}\rangle$. A similar qualitative behavior is seen for the initial condition including $\ket{\rho'}\rangle$, but for a different reason. This decay channel is not of exponential nature, since a factor of $te^{\bar\lambda t}$ enters in the decay. The difference between the two latter decays is not 

\end{document}
