\documentclass[../main.tex]{subfiles}
\graphicspath{{\subfix{../figures/}}}

\begin{document}
\section{Spectrum}
The first step towards simulating the dynamics of the parallel quantum dot system was to calculate the matrix representation of the Liouvillian. The numerical calculations were done in Python using standard tools from the \verb+numpy+ and \verb+scipy+ packages. An already implemented PERLind approach was used to calculate the four jump operators $\hat J_i$, $i\in\{1,2,3,4\}$, each associated with one of the tunneling processes of the system (should I explain which jump operator corresponds with what tunneling process?). The total Liouvillian could then be constructed and compared with the Python-based QmeQ package \cite{qmeq} as a sanity check. The basis and ordering used for $L$ and $\ket{\rho}\rangle$ is given by \cref{eq:basis,eq:order}. Then, the parameter space was searched for exceptional points by numerically calculating the eigenvalues of $L$ for varying $\delta\epsilon$ and $\delta\Gamma$. A degeneracy of eigenvalues was found at $\lambda_5 = \lambda_6\approx -0.5\Gamma$, for $\delta\Gamma = 10^{-6}\Gamma$ and $\delta\epsilon \approx 0.3\Gamma$, see \cref{fig:tuning}. The corresponding eigenvectors, $\ket{\rho_5}\rangle$ and $\ket{\rho_6}\rangle$, was found to also coalesce, confirming the existence of an exceptional point. The eigenvalue and the left and right eigenvector corresponding to the EP will be denoted by $\bar \lambda$, $\langle\bra{\bar\sigma}$, and $\ket{\bar\rho}\rangle$ respectively.
\begin{figure}[H]
    \centering
    \includegraphics[width=0.9\linewidth]{figures/tuning.png}
    \caption{The real and imaginary part of two eigenvalues to $L$, $\lambda_5$ and $\lambda_6$, for varying $\delta\epsilon$. An eigenvalue degeneracy at can be seen at $\delta\epsilon\approx0.3\Gamma$.}
    \label{fig:tuning}
\end{figure}

The full spectrum of the Liouvillian at the exceptional point is given in \cref{fig:spec}, where the eigenvalue degeneracy between $\lambda_5$ and $\lambda_6$ is clear. There is almost a degeneracy between $\lambda_3$ and $\lambda_4$, however, this is not a second EP since the eigenvectors are not parallel. Furthermore, note that all $\lambda_{i>1}$ are on the negative real axis, indicating non-oscillatory, exponential decay towards the steady-state in the dynamics of the system. This steady-state is given by the eigenvector corresponding to the zero eigenvalue $\lambda_1$, since then $\diff{}{t}\ket{\rho}\rangle = 0$.
\begin{figure}[H]
    \centering
    \includegraphics[width=0.8\linewidth]{figures/spectrum.png}
    \caption{The spectrum of the Liouvillian at the EP, where the degeneracy between $\lambda_5$ and $\lambda_6$ is the relevant one. The crosses are used to distinguish eigenvalues which are close to each other.}
    \label{fig:spec}
\end{figure}

\section{Dynamics}

Using \cref{eq:genmode}, the evolution of the system can be given in terms of the eigenvalues and generalized eigenvectors of $L$. Away from an exceptional point, the Liouvillian is diagonalizable, and the terms are purely exponential. The evolution of the density operator is then given by
\begin{equation}\label{eq:dynnonep}
    \ket{\rho(t)}\rangle = \ket{\rho_{ss}}\rangle + \sum_{i=2}^6 c_i e^{\lambda_i t} \ket{\rho_i}\rangle
\end{equation}
where $\ket{\rho_i}\rangle$ is the $i$th right eigenvector of $L$, and $c_i = \langle\braket{\sigma_i|\rho(0)}\rangle$. Here, $\langle\bra{\sigma_i}$ is the $i$th left eigenvector, constructed biorthogonally to $\ket{\rho_i}\rangle$ such that $\langle\braket{\sigma_i|\rho_j}\rangle = \delta_{ij}$, and $\ket{\rho_{ss}}\rangle = c_1 \ket{\rho_1}\rangle$ is the steady-state of the system. Furthermore, $\ket{\rho_{ss}}\rangle$ is the only eigenvector with unity trace, which implies that the rest of the other eigenvectors must be traceless for $\ket{\rho}\rangle$ to have unity trace. The eigenvectors $\ket{\rho_{i>1}}\rangle$ do therefore not describe physical states on their own.

At the exceptional point, the Jordan form of $L$ and its exponential $\exp{(Jt)}$ have to be evaluated. Since the EP is of order two, this results in a $2\times2$ Jordan block. This is the only EP, which means that rest of the Jordan form is diagonal. Using \cref{eq:jordan,eq:expjordan}, this results in the following Jordan form and Jordan exponential:

\begin{equation}
    J = \begin{bmatrix} 0 & 0 & 0 & 0 & 0 & 0 \\
                        0 & \lambda_2 & 0 & 0 & 0 & 0 \\
                        0 & 0 & \lambda_3 & 0 & 0 & 0 \\
                        0 & 0 & 0 & \lambda_4 & 0 & 0 \\
                        0 & 0 & 0 & 0 & \bar \lambda & 1 \\
                        0 & 0 & 0 & 0 & 0 & \bar \lambda \\ \end{bmatrix}, \; 
        e^{Jt} = \begin{bmatrix} 1 & 0 & 0 & 0 & 0 & 0 \\
            0 & e^{\lambda_2t} & 0 & 0 & 0 & 0 \\
            0 & 0 & e^{\lambda_3t} & 0 & 0 & 0 \\
            0 & 0 & 0 & e^{\lambda_4t} & 0 & 0 \\
            0 & 0 & 0 & 0 & e^{\bar \lambda t} & t \\
        0 & 0 & 0 & 0 & 0 & e^{\bar \lambda t} \\ \end{bmatrix},
\end{equation}
since $\lambda_1 = 0$.

Furthermore, the Jordan chain vector, here written as $\ket{\rho'}\rangle$, defined by\\${(L - \bar\lambda I)\ket{\rho'}\rangle = \ket{\bar\rho}\rangle}$ and the corresponding left generalized eigenvector $\langle\bra{\sigma'}$ such that $\langle\braket{\sigma'|\rho'}\rangle = 1$, need to be evaluated. Inserting these vectors into \cref{eq:genmode} leads, after a bit of work, to
\begin{equation}\label{eq:dynep}
    \begin{aligned}
        \ket{\rho(t)}\rangle = \ket{\rho_{ss}}\rangle + \sum_{i=2}^4 c_i e^{\lambda_i t} \ket{\rho_i}\rangle  
                                + (\bar c + c't)e^{\bar \lambda t}\ket{\bar \rho}\rangle + c'e^{\bar\lambda t}\ket{\rho'}\rangle,
    \end{aligned}
\end{equation}
where $\bar c = \langle\braket{\bar\sigma|\rho(0)}\rangle$, and $c' = \langle\braket{\sigma'|\rho(0)}\rangle$.

By numerically calculating the generalized eigenvectors as described in \cref{sec:jordan}, the dynamics at non-EP and at EP given by \cref{eq:dynnonep,eq:dynep}, were implemented. The results were then compared with a numerical ODE-solver, see \cref{fig:minevsint}.

\begin{figure}[H]
    \centering
    \includegraphics[width=0.6\linewidth]{figures/minevsint.png}
    \caption{The relative difference $\mathcal{D}(\rho, \rho_\text{int}) = ||\rho - \rho_\text{int}||_1/||\rho_\text{int}||_1$ between the density matrices calculated with the implemented methods and with a numerical solver. The two lines represent the distances at the EP and away from the EP, using the corresponding implemented methods for each case. The numerical solver was set to an absolute and relative tolerance of $10^{-10}$ and $10^{-6}$ respectively.}
    \label{fig:minevsint}
\end{figure}
The relative difference to the numerical solver is on the order of $10^{-8}$, and with the use of strict tolerances of the numerical solver, this indicates an accurate method. Furthermore, when reducing the tolerances even more, the distance kept lowering, indicating that the numerical solver might be the limiting factor. It is hence clear that the implemented methods produce an accurate result at and away from the exceptional point. 

With a clear picture of the analytical dynamics and numerically calculated eigenvalues and generalized eigenvectors at hand, simulations of the parallel dot system could be done. Firstly, the evolution in the generalized modes was investigated. This was done by introducing initial conditions consisting of linear combinations of the generalized eigenvectors
\begin{equation}\label{eq:ini}
    \ket{\rho(0)}\rangle = \ket{\rho_{ss}}\rangle + \sum_{i=2}^4 b_i\ket{\rho_i}\rangle + \bar b \ket{\bar\rho}\rangle + b'\ket{\rho'}\rangle.
\end{equation}
This way, the constants $c$ in \cref{eq:dynep} are the same as the corresponding constants labeled with $b$ in the initial condition. The generalized modes included in the evolution can then be controlled by the initial condition. To see this, simulations of the density matrix were done and then compared with the steady-state of the system, see \cref{fig:rhodiffrho0}. The particular initial conditions used for each simulation is given in~Table \ref{table:tab}.
\begin{figure}[H]
\centering
\begin{subfigure}[t]{.5\textwidth}
  \centering
  \includegraphics[width=\linewidth]{figures/rho_diff_rho0_v4.png}
  \caption{}
  \label{fig:rhodiffrho0}
\end{subfigure}%
\begin{subfigure}[t]{.5\textwidth}
  \centering
  \includegraphics[width=\linewidth]{figures/I_diff_rho0_nonvis.png}
  \caption{}
  \label{fig:Idiffrho0}
\end{subfigure}
\caption{The decay of the density matrix (a) and the current (b), towards the steady-state density matrix $\rho_{ss}$ and current $I_{ss}$ in a logarithmic plot. The simulations were done at the EP with different initial conditions in linear combinations of the generalized eigenvectors given by Table \ref{table:tab} and \cref{eq:ini}. A normalization by $N=||\rho(0) - \rho_{ss}||$ and $N=||I(0) - I_{ss}||$ was done such that each curve is unity for $t=0$.}
\label{fig:diffrho}
\end{figure}

\begin{table}[H]
    \centering
    \caption{The initial conditions used for the simulations in \cref{fig:diffrho}, in terms of the constants labeled $b$ in \cref{eq:ini}. Only the non-zero constants are included in the table.}
    \begin{tabular}{c|c}\label{table:tab}
        Line & Constants \\\hline 
        Solid, blue & $\bar b = 1$ \\
        Dashed, yellow & $b'=1$ \\
        Solid, green & $b_3 = 1$ \\
        Dash-dotted, red & $\bar b = 1, b_3 = 15$ \\
    \end{tabular}
\end{table}
With initial conditions including only one $b_i\neq0$ or $\bar b\neq0$, i.e., only one normal eigenvector excluding $\ket{\rho_{ss}}\rangle$, a mode consisting of pure exponential decay toward the steady-state is seen by the straight lines in the logarithmic plot. This is to be expected since the time evolution only picks up one of the exponentials from \cref{eq:dynep}. The quicker decay corresponds to $\ket{\rho_3}\rangle$ since $|\lambda_3| > |\bar\lambda|$, see \cref{fig:spec}. When including both $\ket{\bar \rho}\rangle$ and $\ket{\rho_3}\rangle$ in the initial condition, the decay first follows the quicker decay channel, then gradually turns into following the decay of $\ket{\bar\rho}\rangle$. Another behavior is seen for the initial condition including $\ket{\rho'}\rangle$. This generalized mode is not of exponential nature, since a factor of $t\exp(\bar\lambda t)$ enters in the dynamics. Away from an EP, the dynamics is given by \cref{eq:dynnonep} and this type of algebraic decay never happens. Thus, the behavior of this decay channel is a unique feature of the EP dynamics.

As discussed in the theory section, any relevant observable of the system can be obtained from the density operator. For the parallel dot system, the main observable is the current through the system, which by using \cref{eq:expec} can be obtained by
\begin{equation}\label{eq:current}
    \braket{\hat I}(t) = \tr{(\hat \rho(t)\hat I)},
\end{equation}
for the current operator $\hat I$. The current operator can be phrased in terms of the jump operators $\hat{J}_i$ from the Lindblad equation, which were obtained earlier while calculating the Liouvillian $L$. By constructing the current operator and implementing \cref{eq:current} numerically, the current through the quantum dot system for varying parameters and initial conditions could be simulated. In \cref{fig:Idiffrho0}, the calculations from \cref{fig:rhodiffrho0} were used to also simulate the current for the same initial conditions. The results are similar to the simulations of the density matrix, indicating that the evolution in generalized modes carry over to the transient current dynamics. However, with the initial condition including the two normal eigenvectors, the decay in the quicker decay channel is washed out in this time scale and seemingly only follows the slower decay rate. This can be contrasted with the algebraic decay, which is equally visible in the current dynamics as in the density matrix dynamics.

% \begin{equation}
%     \hat{I}_s(t) = \diff{}{t}\hat{N}_s(t) = -i [\hat H, \hat{N}_s(t)],
% \end{equation}

Simulations were also done to investigate the general behavior of the current at and away from the EP. This is done in \cref{fig:diffI}, for two different initial conditions: one with empty dots, and the other in a mixed state. These density matrices are given by
\begin{equation}\label{eq:initialI}
\begin{split}
    1: \rho(0) = \rho_\text{empty} = &\begin{bmatrix} 1 & 0 & 0 & 0 \\
                                             0 & 0 & 0 & 0 \\
                                             0 & 0 & 0 & 0 \\
                                         0 & 0 & 0 & 0 \end{bmatrix},\\ \\
        2: \rho(0) = \rho_\text{mixed} = 1/4 &\begin{bmatrix} 1 & 0 & 0 & 0 \\
                                             0 & 1 & 0 & 0 \\
                                             0 & 0 & 1 & 0 \\
                                            0 & 0 & 0 & 1 \end{bmatrix}.
\end{split}
\end{equation}

\begin{figure}[H]
\centering
\begin{subfigure}[t]{.5\textwidth}
  \centering
  \includegraphics[width=\linewidth]{figures/diffde_empty.png}
  \caption{}
  \label{fig:diffde1}
\end{subfigure}%
\begin{subfigure}[t]{.5\textwidth}
  \centering
  \includegraphics[width=\linewidth]{figures/diffde_mixed.png}
  \caption{}
  \label{fig:diffde2}
\end{subfigure}
\caption{The current over time for different $\Delta = (\delta\epsilon_\text{EP} - \delta\epsilon)$, normalized by the steady-state current $I_{ss}$. The solid, blue curves correspond to the system being at the EP, while the other two are slightly away from it. The initial conditions are given by \cref{eq:initialI}, with a) having the empty initial condition and b) the mixed one.}
\label{fig:diffI}
\end{figure}
Furthermore, it can be noted that the current dynamics in \cref{fig:diffI} varies continuously when approaching and leaving the EP, which is to be expected even though the notion of EPs at first sight seems to invoke abrupt changes. However, there should be no discontinuities in a physical system.   

\end{document}
