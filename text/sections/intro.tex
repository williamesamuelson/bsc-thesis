\documentclass[../main.tex]{subfiles}
\graphicspath{{\subfix{../figs/}}}
\begin{document}
The Hermiticity of the Hamiltonian in quantum mechanics is thought of being a necessity for physical behavior of a quantum system. However, in recent decades, the field of non-Hermitian quantum physics has shown major progress, both theoretically and experimentally. Systems of an "open nature", i.e systems being connected to an environment, in nuclear, atomic and optical physics have been accurately described by non-Hermitian Hamiltonians~\cite{nonHermrev}. One of the main properties of non-Hermitian operators is the possibility of exceptional points (EPs). These correspond to points in parameter space where two or more eigenvalues and their corresponding eigenvectors of the operator simultaneously coalesce. Exceptional points have been proposed to have several useful technical applications, and along with the optical microring experiments in 2017, exceptional point sensors successfully increased the sensitivity of current and nano-particle detection~\cite{microring1, microring2}.

Another operator in quantum mechanics which more recently has been studied for its native non-Hermiticity is the Liouvillian superoperator~\cite{recentliou, thermal, steering}. The Liouvillian and its corresponding quantum master equation describe the dynamics in open quantum systems, where the system is coupled to a larger environment. Due to dissipation to the environment in such systems, the Liouvillian is not necessarily Hermitian, which again brings the possibility of exceptional points. EPs in Liouvillian physics affect the dynamics of the density operator, and therefore the evolution of all relevant observables in the system. Recently, exceptional points in the Liouvillian have been theoretically studied in quantum thermal machines, where critical decay towards the steady state was found at the EP~\cite{thermal}; and in quantum control, where an EP corresponded to optimal steering toward a target quantum state~\cite{steering}. 

Another application of interest for quantum master equations and Liouvillian physics is electron transport in systems of quantum dots connected to metallic leads~\cite{qdottrans}. A quantum dot is a fabricated semiconductor structure containing a small number of electrons and is typically in the order of 100s nanometres in size (?). (something about the thermal wavelength) One way of creating this isolation of electrons is to apply voltages via nanoscale electrodes, called gates, which depletes the number of electrons in the region close to the tip of the gates. If tuned successfully, the quantum dot can be tunnel coupled to the surrounding conducting regions of the semiconductor, known as the source and drain. The electrons can then tunnel into and out of the quantum dot, producing a current through the system. (sources and figure)

The dynamics of the current through a quantum dot system can be derived from the microscopic properties of the system, resulting in a quantum master equation and a corresponding Liouvillian. The matrix representation of the Liouvillian will depend on the tunnel coupling strength between the dot and the source and drain, the gate and drain voltages, and all other parameters which describe the system. Since the Liouvillian is not necessarily Hermitian, it is possible to find points in this parameter space which correspond to an exceptional point.

The system considered in this thesis consists of two quantum dots coupled to two metallic leads, corresponding to the source and drain. The dots are assumed to be coupled in parallel, i.e not directly coupled to each other but only indirectly through the two leads, see figure X. It is also assumed that each dot is restricted to be either empty or contain one electron, which reduces the dimension of the problem to a (?) degree. Furthermore, the coupling between the dots and the leads are assumed to have two properties: that it can be described by a Markovian interaction, and that it is weak. (reasonable assumptions?) These last assumptions enables the use of the Lindblad master equation to calculate the Liouvillian operator of the system.


\end{document}

