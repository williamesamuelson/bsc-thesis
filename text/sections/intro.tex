\documentclass[../main.tex]{subfiles}
\graphicspath{{\subfix{../figs/}}}
\begin{document}
The Hermiticity of the Hamiltonian in quantum mechanics has historically been thought of a necessity for physical behavior of a quantum system (?). However, in recent (what), the field of non-Hermitian quantum mechanics has shown major progress, both theoretically and experimentally. Systems in cavity physics, optics, (something) have been shown to be well described by non-Hermitian Hamiltonians. One of the main properties of non-Hermitian operators is the possibility of exceptional points. These correspond to points in parameter space where two or more eigenvalues and their corresponding eigenvectors of the operator simultaneously coalesce. Exceptional points have been proposed 

Another quantum mechanical operator which more recently has been studied for its (native) non-Hermiticity is the Liouvillian superoperator. This operator describes the evolution of the density operator for systems connected to an environment. One example of such a system are quantum dot systems coupled to metallic leads.





%Solid-state devices based on quantum dots are of much interest in recent research. Not only used as an experimental playground, but these devices also have potential in future technologies such as quantum computation components and delicate sensing devices. The notion of transport through quantum dots 


\end{document}

