\documentclass[../main.tex]{subfiles}
\graphicspath{{\subfix{../figures/}}}
\begin{document}
\thispagestyle{plain}
\begin{center}
    \Large
    \textbf{Dynamics at an exceptional point in an interacting quantum dot system}
    
    \vspace{0.4cm}
    
    \vspace{0.4cm}
    
    \vspace{0.9cm}
    \textbf{Abstract}
\end{center}
A fundamental postulate of quantum mechanics is that the Hamiltonian of a closed system is Hermitian. This guarantees real-valued energies and conserved probabilities throughout the evolution of the system. However, for quantum systems where there is dissipation in and out of the system, such as a collection of quantum dots connected to metallic leads, the dynamics is instead generated by a non-Hermitian Liouvillian superoperator. One feature of non-Hermitian operators with particular theoretical and experimental interest recently, is the possibility of exceptional points. These are points in parameter space which causes two or more eigenvalues and their corresponding eigenvectors of the operator to simultaneously coalesce. Exceptional points have been proposed to have several technological applications and successfully increased the sensitivity of sensors. In this thesis, we analyze a system of two quantum dots coupled in parallel to metallic leads and demonstrate the existence of a second order exceptional point in the Liouvillian superoperator. Furthermore, the dynamics at this exceptional point is simulated using a combination of analytical and numerical methods, including simulations of the density operator and the current through the system. We show that the dynamics can be understood in terms of generalized modes and that the system has a unique algebraic decay at the exceptional point. Furthermore, critical decay at the exceptional point is indicated in the current, in accordance with other works on exceptional points in quantum thermal machines.
\end{document}
