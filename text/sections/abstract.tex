\documentclass[../main.tex]{subfiles}
\graphicspath{{\subfix{../figures/}}}
\begin{document}
\thispagestyle{plain}
\begin{center}
    \Large
    \textbf{Dynamics at an exceptional point in an interacting quantum dot system}
    
    \vspace{0.4cm}
    
    \vspace{0.4cm}
    
    \vspace{0.9cm}
    \textbf{Abstract}
\end{center}
A fundamental postulate of quantum mechanics is that the Hamiltonian of a closed system is Hermitian, guaranteeing real-valued energies and conserved probabilities throughout the evolution of the system. However, for quantum systems where there is dissipation and crosstalk with the environment, such as a system of quantum dots connected to an environment of metallic leads, the dynamics of the system is instead generated by a non-Hermitian Liouvillian superoperator. One feature of non-Hermitian operators, showing increasing theoretical and experimental interest recently, is the possibility of exceptional points. These are points in parameter space which cause two or more eigenvalues and their corresponding eigenvectors of the operator to simultaneously coalesce. Exceptional points can be of different orders, corresponding to the number of coalescing eigenvectors and eigenvalues. In this thesis, we study a system of two quantum dots coupled in parallel to metallic leads and demonstrate the existence of a second order exceptional point in the Liouvillian superoperator. Furthermore, the dynamics at this exceptional point is analyzed in detail using a combination of analytical and numerical methods, including simulations of the density operator and the current through the system. We show that the dynamics can be understood in terms of generalized modes and that the system exhibits unique algebraic decay at the exceptional point. Furthermore, critical damping at the exceptional point is indicated in the current, in accordance with a previous work on exceptional points in quantum thermal machines. We also briefly discuss possible applications of exceptional points in quantum dots, including control and sensing technologies. \end{document}
